%%%%%%%%%%%%%%%%%%%%%%%%%%%%%%%%%%%%%%%%%
% The Legrand Orange Book
% LaTeX Template
% Version 2.1.1 (14/2/16)
%
% This template has been downloaded from:
% http://www.LaTeXTemplates.com
%
% Original author:
% Mathias Legrand (legrand.mathias@gmail.com) with modifications by:
% Vel (vel@latextemplates.com)
%
% License:
% CC BY-NC-SA 3.0 (http://creativecommons.org/licenses/by-nc-sa/3.0/)
%
% Compiling this template:
% This template uses biber for its bibliography and makeindex for its index.
% When you first open the template, compile it from the command line with the
% commands below to make sure your LaTeX distribution is configured correctly:
%
% 1) pdflatex main
% 2) makeindex main.idx -s StyleInd.ist
% 3) biber main
% 4) pdflatex main x 2
%
% After this, when you wish to update the bibliography/index use the appropriate
% command above and make sure to compile with pdflatex several times
% afterwards to propagate your changes to the document.
%
% This template also uses a number of packages which may need to be
% updated to the newest versions for the template to compile. It is strongly
% recommended you update your LaTeX distribution if you have any
% compilation errors.
%
% Important note:
% Chapter heading images should have a 2:1 width:height ratio,
% e.g. 920px width and 460px height.
%
%%%%%%%%%%%%%%%%%%%%%%%%%%%%%%%%%%%%%%%%%

%----------------------------------------------------------------------------------------
%	PACKAGES AND OTHER DOCUMENT CONFIGURATIONS
%----------------------------------------------------------------------------------------

\documentclass[11pt,fleqn]{book} % Default font size and left-justified equations

%----------------------------------------------------------------------------------------

\input{structure} % Insert the commands.tex file which contains the majority of the structure behind the template

\usepackage{amsmath,amssymb,wasysym}



\begin{document}

%----------------------------------------------------------------------------------------
%	TITLE PAGE
%----------------------------------------------------------------------------------------

\begingroup
\thispagestyle{empty}
\begin{tikzpicture}[remember picture,overlay]
\coordinate [below=12cm] (midpoint) at (current page.north);
\node at (current page.north west)
{\begin{tikzpicture}[remember picture,overlay]
\node[anchor=north west,inner sep=0pt] at (0,0) {\includegraphics[width=\paperwidth]{background}}; % Background image
\draw[anchor=north] (midpoint) node [fill=ocre!30!white,fill opacity=0.6,text opacity=1,inner sep=1cm]{\Huge\centering\bfseries\sffamily\parbox[c][][t]{\paperwidth}{\centering \textsc{Notes of Microeconomic Theory}\\[15pt] % Book title
{\Large By Mas-Colell Whinston And Green}\\[20pt] % Subtitle
{\huge Huyi Chen}}}; % Author name
\end{tikzpicture}};
\end{tikzpicture}
\vfill
\endgroup

%----------------------------------------------------------------------------------------
%	COPYRIGHT PAGE
%----------------------------------------------------------------------------------------

\newpage
~\vfill
\thispagestyle{empty}

\noindent Copyright \copyright\ 2017 Huyi Chen\\ % Copyright notice

\noindent \textsc{Published by Publisher}\\ % Publisher

\noindent \textsc{book-website.com}\\ % URL

\noindent Licensed under the Creative Commons Attribution-NonCommercial 3.0 Unported License (the ``License''). You may not use this file except in compliance with the License. You may obtain a copy of the License at \url{http://creativecommons.org/licenses/by-nc/3.0}. Unless required by applicable law or agreed to in writing, software distributed under the License is distributed on an \textsc{``as is'' basis, without warranties or conditions of any kind}, either express or implied. See the License for the specific language governing permissions and limitations under the License.\\ % License information

\noindent \textit{First printing, March 2013} % Printing/edition date

%----------------------------------------------------------------------------------------
%	TABLE OF CONTENTS
%----------------------------------------------------------------------------------------

%\usechapterimagefalse % If you don't want to include a chapter image, use this to toggle images off - it can be enabled later with \usechapterimagetrue

\chapterimage{contents.pdf} % Table of contents heading image

\pagestyle{empty} % No headers

\tableofcontents % Print the table of contents itself

\cleardoublepage % Forces the first chapter to start on an odd page so it's on the right

\pagestyle{fancy} % Print headers again

%----------------------------------------------------------------------------------------
%	PART
%----------------------------------------------------------------------------------------
\part{Part One: Individual Decision Making}

%----------------------------------------------------------------------------------------
%	CHAPTER 1
%----------------------------------------------------------------------------------------

\chapterimage{chapter_head_1.pdf} % Chapter heading image

\chapter{Preference and Choice}

\section{Preference-based Approach}

\begin{definition}[Preference Relation]
	Preference relation $\succsim$ is a binary relation defined on the set of alternatives $X$.
\end{definition}

\begin{definition}[Rational]
	The preference relation $\succsim$ is \emph{rational} if it possesses the following two properties:
	\begin{enumerate}
		\item Completeness: $\forall x,y\in X,\;x\succsim y$ or $y\succsim x$.
		\item Transitivity: $\forall x,y,z\in X,\;x\succsim y$ and $y\succsim z \implies x\succsim z$.
	\end{enumerate}
\end{definition}

\begin{definition}[Strict Preference Relation]
	Define the \emph{strict preference relation} $\succ$ as follows:
	\[	
	x\succ y \Longleftrightarrow x\succsim y \ \text{but not}\  y\succsim x.
	\]
\end{definition}

\begin{definition}[Indeference Relation]
	Define the \emph{indifference relation} $\sim$ as follows:
	\[	
	x\sim y \Longleftrightarrow x\succsim y \ \text{and}\  y\succsim x.
	\]
\end{definition}

\begin{definition}[Utility Function]
    A function $u:X\rightarrow\mathbb{R}$ is a \emph{utility function representing preference relation} $\succsim$ if
    \[
    \forall x,y\in X,\;x\succ y\Longleftrightarrow u(x)\ge u(y).
    \]
\end{definition}

\begin{proposition}
	If there is a utility function that represents a preference relation $\succsim$, then $\succsim$ must be rational. 
\end{proposition}


\section{Choice-based Approach}

\begin{definition}[Choice Structure]
	A \emph{choice structure} is a binary tuple $(\mathcal{B},C(\cdot))$, where the set $\mathcal{B}\subset 2^X\slash \{\varnothing\}$ is a family of nonempty subsets of $X$ and the mapping $C:\mathcal{B}\rightarrow \mathcal{B}$ assigns a nonempty set of chosen elements $C(B)\subset B$ for every budget set $B\in\mathcal{B}$.
\end{definition}

\begin{definition}[Revealed Preference Relation]
	Given a choice structure $(\mathcal{B},C(\cdot))$ the \emph{revealed preference relation} $\succsim^*$ is a binary relation on $X$ defined by
	\[
	x\succsim^*y\Longleftrightarrow \exists B\in\mathcal{B},\;x,y\in B \ \text{and}\  x\in C(B).
	\]
\end{definition}

\begin{remark}
	For convenience we define a binary relation $\succ^*$ on $X$ informally as follows 
	\[
	x\succ^*y\Longleftrightarrow \exists B\in\mathcal{B},\;x,y\in B \ \text{and}\  x\in C(B) \ \text{and}\ y\notin C(B).
	\]
	We will say "$x$ is revealed preferred to $y$" if $x\succ^*y$.
\end{remark}

\begin{definition}[Weak Axiom of Revealed Preference]
	The choice structure $(\mathcal{B},C(\cdot))$ satisfies the \emph{weak axiom of revealed preference} if the following property
	\[
	B,B'\in \mathcal{B}\ \text{and}\ x,y\in B\ \text{and}\ x,y\in B'\ \text{and}\ x\in C(B)\ \text{and}\ y\in C(B')\implies x\in C(B')
	\]
	or equivalently
	\[
	\forall x,y\in X,\;x\succsim^*y\implies\text{not}\ y\succ^*x
	\]
	holds.
\end{definition}

\section{The Relationship between Preference Relations and Choice Rules }

\subsection{Rationality Implies WARP}
\vspace{4pt}
\begin{definition}
	Define the correspondence $C^*(\cdot,\succsim):\mathcal{B}\rightarrow 2^X$ as 
	\[
	C^*(B,\succsim)=\{x\in B\;|\;\forall y\in B, x\succsim y\}.
	\]
	We say the preference $\succsim$ \emph{generates} the choice structure $(\mathcal{B},C^*(\cdot,\succsim))$ if $C^*(B,\succsim)\ne\varnothing$ for all $B\in\mathcal{B}$.
	
\end{definition}

\begin{remark}
    If $X$ is finite, then $C^*(B,\succsim)$ will be nonempty. From now on, we will consider only preferences $\succsim$ and families of budget sets $\mathcal{B}$ such that $C^*(B,\succsim)$ is nonempty for all $B\in\mathcal{B}$.
\end{remark}

\begin{proposition}[Rationality Implies WARP]
	Suppose that $\succsim$ is a rational preference relation. Then the choice structure generated by $\succsim$, $(\mathcal{B},C^*(\cdot,\succsim))$, satisfies the weak axiom of revealed preference. 
\end{proposition}

\subsection{WARP does NOT Implies Rationality}
\vspace{4pt}
\begin{definition}[Rationalization]
	Given a choice structure $(\mathcal{B},C(\cdot))$, we say that the rational preference	relation $\succsim$ \emph{rationalizes} $C(\cdot)$ relative to $\mathcal{B}$ if
	\[
	\forall B\in\mathcal{B},\ C^*(B,\succsim)=C(B),
	\]
	that is, if the choice structure $(\mathcal{B},C^*(\cdot,\succsim))$  generated by $\succsim$ is identical with $(\mathcal{B},C(\cdot))$.
\end{definition}

\begin{proposition}
	If $(\mathcal{B},C(\cdot))$ is a choice structure such that
	\begin{enumerate}
		\item[(\romannumeral1)] the  weak axiom is satisfied,
		\item[(\romannumeral2)] $\mathcal{B}$ includes all subsets of $X$ of up to three elements,
	\end{enumerate}
	then there is a rational preference relation $\succsim$ that rationalizes $C(\cdot)$ relative to $\mathcal{B}$; that is, $C^*(B,\succsim)=C(B)$ for all $B\in\mathcal{B}$. Furthermore, this rational preference relation is the only preference relation that does so.
\end{proposition}

\chapterimage{chapter_head_2.pdf} % Chapter heading image

\chapter{Consumer Choice}

\section{Preference-based Approach}


\chapterimage{chapter_head_3.pdf}

\chapter{Classical Demand Theory}

\section{Preference Relation: Basic Properties}

To start with, let's introduce some notations for convenience.
Supposing that $x,y\in\mathbb{R}_+^L$, then $x$ and $y$ can have the following relations:
\begin{enumerate}
	\item $y\gg x \Longleftrightarrow y_i>x_i\;(i=1,2,\cdots,L);$
	\item $y\geqslant x \Longleftrightarrow y_i\geqslant x_i\;(i=1,2,\cdots,L);$
	\item $y>x\Longleftrightarrow y\geqslant x$ and $y\ne x.$
\end{enumerate}

Given a preference relation $\succsim$ on consumption set $X$, these sets are common to meet:
\begin{enumerate}
	\item upper contour set: $R(x)=\{y\in X|\,y\succsim x\};$
	\item lower contour set: $R^{-1}(x)=\{y\in X|\,x\succsim y\};$
	\item $P(x)=\{y\in X|\,y\succ x\}$;
	\item $P^{-1}(x)=\{y\in X|\,x\succ y\}$.
\end{enumerate}

We assume throughout that the preference relation $\succsim$ is rational in the sense introduced in Section 1.1, that is, $\succsim$ is complete and transitive.

\begin{definition}[Local Nonsatiation]
	A preference relation $\succsim$ on $X$ is \emph{locally nonsatiated} if
	\[
	\forall x\in X,\;\forall \varepsilon>0,\;\exists y\in X,\;\|y-x\|\le\varepsilon\;\text{and}\;y\succ x
	\]
	or equivalently
	\[
	\forall x\in X,\;\forall \varepsilon>0,\;B(x,\epsilon)\cap P(x)\ne\varnothing.
	\]
\end{definition}

\begin{definition}[Monotonicity]
	A preference relation $\succsim$ on $X$ is \emph{monotone} if
	\[
	\forall x,y\in X,\;y\gg x\implies y\succ x
	\]
	or equivalently
	\[
	\forall x\in X,\;\{y\in X|\,y\gg x\}\subset P(x).
	\]
\end{definition}

\begin{definition}[Strong Monotonicity]
	A preference relation $\succsim$ on $X$ is \emph{strongly monotone} if
	\[
	\forall x,y\in X,\;y\geqslant x\;\text{and}\;y\ne x\implies y\succ x
	\]
	or equivalently
	\[
	\forall x\in X,\;\{y\in X|\,y> x\}\subset P(x).
	\]
\end{definition}

\begin{proposition}
	Let $\succsim$ be a preference relationon on $X$.
	\begin{enumerate}	
		\item If $\succsim$ is strongly monotone, then it is monotone.	
		\item If $\succsim$ is monotone, then it is locally nonsatiated.		 
	\end{enumerate}
\end{proposition}

\begin{definition}[Convexity]
	A preference relation $\succsim$ on $X$ is \emph{convex} if for all $x,y,z\in X$,
	\[
    y\succsim x\;\text{and}\;z\succsim x\implies \forall\alpha\in[0,1],\;\alpha y+(1-\alpha)z\succsim x
    \]
	or equivalently
	\[
	\forall x\in X,\;R(x)\;\text{is a convex set}.
	\]
\end{definition}

\begin{definition}[Strict Convexity]
	A preference relation $\succsim$ on $X$ is \emph{strictly convex} if for all $x,y,z\in X$,
	\[
	y\succsim x\;\text{and}\;z\succsim x\;\text{and}\;y\ne z\implies \forall\alpha\in(0,1),\;\alpha y+(1-\alpha)z\succ x
	\]
	or equivalently for all $x\in X$,
	\[
	\forall y,z\in R(x),\;y\ne z\implies\forall\alpha\in(0,1),\;\alpha y+(1-\alpha)z\in P(x).
	\]
\end{definition}

\begin{proposition}
	Let $\succsim$ be a preference relationon on $X$. If $\succsim$ is strictly convex, then it is convex.	
\end{proposition}

\begin{definition}[Homothetic Preference]
    A monotone preference relation $\succsim$ on $X=\mathbb{R}_+^L$ is \emph{homothetic} if for all $x,y\in X$,
    \[
    x\sim y \implies \forall \alpha\geqslant 0,\;\alpha x\sim\alpha y.
    \]
\end{definition}

\begin{definition}[Quasilinear Preference]
	The preference relation $\succsim$ on $X=(-\infty,+\infty)\times\mathbb{R}_+^{L-1}$ is \emph{quasilinear}
	with respect to commodity 1 (called, in this case, the numeraire commodity) if
	\begin{enumerate}
		\item All the indifference sets are parallel displacements of each other along the	axis of commodity 1. That is, if $x \sim y$, then $(x + \alpha e_1) \sim (y + \alpha e_1)$ for $e_1=(1,0,\cdots,0)$ and any $\alpha\in\mathbb{R}$.
		\item Commodity 1 is desirable; that is, $x +\alpha e\succ x$ for all $x\in X$ and $a > 0$.
	\end{enumerate}
\end{definition}


\section{Preference and Utility}

\begin{definition}[Continuity]
    The preference relation $\succsim$ on $X$ is \emph{continuous} if it is preserved under limits, that is, for any sequence of pairs $\{(x_n, y_n)\}_{n=1}^\infty$ with $x_n\succsim y_n$ for all $n\in\mathbb{N^*}$,
    \begin{equation*}
    \left.
    \begin{aligned}
    	\lim_{n\to\infty} x_n=x\\
    	\lim_{n\to\infty} y_n=y
    \end{aligned}
    \right\}
    \implies x\succsim y,
    \end{equation*}
    or alternatively
    \[
    \forall x\in X,\; R(x)\;\text{and}\;R^{-1}(x)\;\text{are closed set}.
    \]
    
\end{definition}

\begin{theorem}
	Suppose that the rational preference relation $\succsim$ on $X=\mathbb{R}_+^L$ is continuous.
	Then there is a continuous utility function $u(x)$ that represents $\succsim$.
\end{theorem}

\begin{definition}[Quasiconcavity]
	The utility function $u(x)$ is \emph{quasiconcave} if
	\[
	\forall x\in\mathbb{R}_+^L,\;\{y\in\mathbb{R}_+^L|\,u(y)\geqslant u(x)\}\;\text{is convex}
	\]
	or alternatively
	\[
	\forall x,y\in\mathbb{R}_+^L,\;\forall \alpha\in[0,1],\;u(\alpha x+(1-\alpha)y)\geqslant \min\{u(x),u(y)\}).
	\]
\end{definition}

\begin{proposition}
	The utility function $u(\cdot)$ is quasiconcave if and only if it represents a convex preference.
\end{proposition}


\begin{definition}[Strict Quasiconcavity]
	The utility function $u(x)$ is \emph{strictly quasiconcave} if
	\[
	\forall x\in\mathbb{R}_+^L,\;\{y\in\mathbb{R}_+^L|\,u(y)\geqslant u(x)\}\;\text{is strictly convex}
	\]
	or alternatively
	\[
	\forall x,y\in\mathbb{R}_+^L,\;\forall \alpha\in(0,1),\;x\ne y\implies u(\alpha x+(1-\alpha)y)>\min\{u(x),u(y)\}.
	\]
\end{definition}

\begin{proposition}
	The utility function $u(\cdot)$ is strictly quasiconcave if and only if it represents a strictly convex preference.
\end{proposition}

\begin{proposition}
	If the utility function $u(\cdot)$ is strictly quasiconcave, then it is quasiconcave.
\end{proposition}


\section{The Utility Maximization Problem}


\part{Part X}

%----------------------------------------------------------------------------------------
%	CHAPTER 1
%----------------------------------------------------------------------------------------

\chapterimage{chapter_head_1.pdf} % Chapter heading image

\chapter{Text Chapter}

\section{Paragraphs of Text}\index{Paragraphs of Text}

\lipsum[1-7] % Dummy text

%------------------------------------------------

\section{Citation}\index{Citation}

This statement requires citation \cite{book_key}; this one is more specific \cite[122]{article_key}.

%------------------------------------------------

\section{Lists}\index{Lists}

Lists are useful to present information in a concise and/or ordered way\footnote{Footnote example...}.

\subsection{Numbered List}\index{Lists!Numbered List}

\begin{enumerate}
\item The first item
\item The second item
\item The third item
\end{enumerate}

\subsection{Bullet Points}\index{Lists!Bullet Points}

\begin{itemize}
\item The first item
\item The second item
\item The third item
\end{itemize}

\subsection{Descriptions and Definitions}\index{Lists!Descriptions and Definitions}

\begin{description}
\item[Name] Description
\item[Word] Definition
\item[Comment] Elaboration
\end{description}

%----------------------------------------------------------------------------------------
%	CHAPTER 2
%----------------------------------------------------------------------------------------
\chapterimage{chapter_head_2.pdf}
\chapter{In-text Elements}

\section{Theorems}\index{Theorems}

This is an example of theorems.

\subsection{Several equations}\index{Theorems!Several Equations}
This is a theorem consisting of several equations.

\begin{theorem}[Name of the theorem]
In $E=\mathbb{R}^n$ all norms are equivalent. It has the properties:
\begin{align}
& \big| ||\mathbf{x}|| - ||\mathbf{y}|| \big|\leq || \mathbf{x}- \mathbf{y}||\\
&  ||\sum_{i=1}^n\mathbf{x}_i||\leq \sum_{i=1}^n||\mathbf{x}_i||\quad\text{where $n$ is a finite integer}
\end{align}
\end{theorem}

\subsection{Single Line}\index{Theorems!Single Line}
This is a theorem consisting of just one line.

\begin{theorem}
A set $\mathcal{D}(G)$ in dense in $L^2(G)$, $|\cdot|_0$.
\end{theorem}

%------------------------------------------------

\section{Definitions}\index{Definitions}

This is an example of a definition. A definition could be mathematical or it could define a concept.

\begin{definition}[Definition name]
Given a vector space $E$, a norm on $E$ is an application, denoted $||\cdot||$, $E$ in $\mathbb{R}^+=[0,+\infty[$ such that:
\begin{align}
& ||\mathbf{x}||=0\ \Rightarrow\ \mathbf{x}=\mathbf{0}\\
& ||\lambda \mathbf{x}||=|\lambda|\cdot ||\mathbf{x}||\\
& ||\mathbf{x}+\mathbf{y}||\leq ||\mathbf{x}||+||\mathbf{y}||
\end{align}
\end{definition}

%------------------------------------------------

\section{Notations}\index{Notations}

\begin{notation}
Given an open subset $G$ of $\mathbb{R}^n$, the set of functions $\varphi$ are:
\begin{enumerate}
\item Bounded support $G$;
\item Infinitely differentiable;
\end{enumerate}
a vector space is denoted by $\mathcal{D}(G)$.
\end{notation}

%------------------------------------------------

\section{Remarks}\index{Remarks}

This is an example of a remark.

\begin{remark}
The concepts presented here are now in conventional employment in mathematics. Vector spaces are taken over the field $\mathbb{K}=\mathbb{R}$, however, established properties are easily extended to $\mathbb{K}=\mathbb{C}$.
\end{remark}

%------------------------------------------------

\section{Corollaries}\index{Corollaries}

This is an example of a corollary.

\begin{corollary}[Corollary name]
The concepts presented here are now in conventional employment in mathematics. Vector spaces are taken over the field $\mathbb{K}=\mathbb{R}$, however, established properties are easily extended to $\mathbb{K}=\mathbb{C}$.
\end{corollary}

%------------------------------------------------

\section{Propositions}\index{Propositions}

This is an example of propositions.

\subsection{Several equations}\index{Propositions!Several Equations}

\begin{proposition}[Proposition name]
It has the properties:
\begin{align}
& \big| ||\mathbf{x}|| - ||\mathbf{y}|| \big|\leq || \mathbf{x}- \mathbf{y}||\\
&  ||\sum_{i=1}^n\mathbf{x}_i||\leq \sum_{i=1}^n||\mathbf{x}_i||\quad\text{where $n$ is a finite integer}
\end{align}
\end{proposition}

\subsection{Single Line}\index{Propositions!Single Line}

\begin{proposition}
Let $f,g\in L^2(G)$; if $\forall \varphi\in\mathcal{D}(G)$, $(f,\varphi)_0=(g,\varphi)_0$ then $f = g$.
\end{proposition}

%------------------------------------------------

\section{Examples}\index{Examples}

This is an example of examples.

\subsection{Equation and Text}\index{Examples!Equation and Text}

\begin{example}
Let $G=\{x\in\mathbb{R}^2:|x|<3\}$ and denoted by: $x^0=(1,1)$; consider the function:
\begin{equation}
f(x)=\left\{\begin{aligned} & \mathrm{e}^{|x|} & & \text{si $|x-x^0|\leq 1/2$}\\
& 0 & & \text{si $|x-x^0|> 1/2$}\end{aligned}\right.
\end{equation}
The function $f$ has bounded support, we can take $A=\{x\in\mathbb{R}^2:|x-x^0|\leq 1/2+\epsilon\}$ for all $\epsilon\in\intoo{0}{5/2-\sqrt{2}}$.
\end{example}

\subsection{Paragraph of Text}\index{Examples!Paragraph of Text}

\begin{example}[Example name]
\lipsum[2]
\end{example}

%------------------------------------------------

\section{Exercises}\index{Exercises}

This is an example of an exercise.

\begin{exercise}
This is a good place to ask a question to test learning progress or further cement ideas into students' minds.
\end{exercise}

%------------------------------------------------

\section{Problems}\index{Problems}

\begin{problem}
What is the average airspeed velocity of an unladen swallow?
\end{problem}

%------------------------------------------------

\section{Vocabulary}\index{Vocabulary}

Define a word to improve a students' vocabulary.

\begin{vocabulary}[Word]
Definition of word.
\end{vocabulary}






%----------------------------------------------------------------------------------------
%	PART
%----------------------------------------------------------------------------------------

%\part{Part Two}

%----------------------------------------------------------------------------------------
%	CHAPTER 3
%----------------------------------------------------------------------------------------

\chapterimage{chapter_head_3.pdf} % Chapter heading image

\chapter{Presenting Information}

\section{Table}\index{Table}

\begin{table}[h]
\centering
\begin{tabular}{l l l}
\toprule
\textbf{Treatments} & \textbf{Response 1} & \textbf{Response 2}\\
\midrule
Treatment 1 & 0.0003262 & 0.562 \\
Treatment 2 & 0.0015681 & 0.910 \\
Treatment 3 & 0.0009271 & 0.296 \\
\bottomrule
\end{tabular}
\caption{Table caption}
\end{table}

%------------------------------------------------

\section{Figure}\index{Figure}

\begin{figure}[h]
\centering\includegraphics[scale=0.5]{placeholder}
\caption{Figure caption}
\end{figure}




%----------------------------------------------------------------------------------------
%	PART
%----------------------------------------------------------------------------------------

\part{Part Two}

%----------------------------------------------------------------------------------------
%	CHAPTER 3
%----------------------------------------------------------------------------------------
\chapterimage{chapter_head_4.pdf} % Chapter heading image

\chapter{Limit Of Sequence}

\section{Cauchy proposition}

\begin{theorem}
    If a sequence $\{x_n\}$ converges to $l$, then its arithmetic mean of the preceding $n$ terms also converges to $l$, namely
    \begin{equation}
    \lim\limits_{n\to\infty}\frac{x_1+x_2+\cdots+x_n}{n}=\lim\limits_{n\to\infty}x_n=l.
    \end{equation}
\end{theorem}

\begin{proof}
	According to the condition $\lim\limits_{n\to\infty}x_n=l$, given $\epsilon>0$, there exists a positive number $N$ such that $|x_n-l|<\epsilon$ for all $n>N$. Assume $n>N$ and in this case we can make an estimation as follows:
	\begin{align*}
	\left|\frac{x_1+x_2+\cdots+x_n}{n}-l\right|
	&=\frac{\left|(x_1-l)+(x_2-l)+\cdots+(x_n-l)\right|}{n}\\
	&\leqslant\frac{\left|(x_1-l)+\cdots+(x_N-l)\right|}{n}+\frac{\left|(x_{N+1}-l)+\cdots+(x_n-l)\right|}{n}\\
	&<\frac{M}{n}+\frac{n-N}{n}\epsilon,
	\end{align*}
    where $M=\left|(x_1-l)+\cdots+(x_N-l)\right|$ is a finite number. Thus we can see that if let
	\[
	N_1=\max\{N,\left[\frac{M}{\epsilon}\right]\},
	\]
	then for all $n>N_1$ it follows that
	\[
		\left|\frac{x_1+x_2+\cdots+x_n}{n}-l\right|<2\epsilon.
	\]
	It clearly implies $
	\lim\limits_{n\to\infty}\dfrac{x_1+x_2+\cdots+x_n}{n}=l.
$

\end{proof}

\begin{remark}
    \begin{enumerate}
    \item
	If $x_n$ approaches positive(or negative) infinity, Cauchy proposition still holds. In fact, from $x_n\to+\infty(n\to\infty)$ we see $x_n$ can be greater than any given positive number $X$ when $n$ is large enough. Similarly we can separate the arithmetic mean into two parts and show the second part $(1-N/n)X$ greater than $X/2$ for a sufficiently large $n$.\\
	\item
	The converse of Cauchy proposition is generally not true. A trivial example is $x_n=(-1)^n$. Then
	\[
	\lim\limits_{n\to\infty}\frac{x_1+x_2+\cdots+x_n}{n}=0
	\]
	while $x_n$ has no limit.
    \end{enumerate}
\end{remark}
\vspace{3mm}
\begin{corollary}
	If a positive term sequence $\{x_n\}$ converges to $l$, then its geometric mean of the preceding $n$ terms also converges to $l$, namely
	\begin{equation}
	\lim\limits_{n\to\infty}\sqrt[^n]{x_1x_2\cdots x_n}=\lim\limits_{n\to\infty}x_n=l.
	\end{equation}
\end{corollary}

\begin{proof}
	Applying the mean inequality we have
    \[
    \frac{n}{\frac{1}{x_1}+\frac{1}{x_2}+\cdots+\frac{1}{x_n}}\leqslant\sqrt[^n]{x_1x_2\cdots x_n}\leqslant\frac{x_1+x_2+\cdots+x_n}{n}.
    \]
    Notice that $\lim\limits_{n\to\infty}x_n=l$ implies $\lim\limits_{n\to\infty}\dfrac{1}{x_n}=\dfrac{1}{l}$. From Theorem 4.1.1 we can see
    \[
    \lim\limits_{n\to\infty}\frac{n}{\frac{1}{x_1}+\frac{1}{x_2}+\cdots+\frac{1}{x_n}}=\lim\limits_{n\to\infty}\left(\frac{\frac{1}{x_1}+\frac{1}{x_2}+\cdots+\frac{1}{x_n}}{n}\right)^{-1}=\left(\lim\limits_{n\to\infty}\frac{\frac{1}{x_1}+\frac{1}{x_2}+\cdots+\frac{1}{x_n}}{n}\right)^{-1}=l.
    \]
    And it has been shown that
    \[
    \lim\limits_{n\to\infty}\frac{x_1+x_2+\cdots+x_n}{n}=l.
    \]
    According to squeeze theorem, we can assert that the limit $\lim\limits_{n\to\infty}\sqrt[^n]{x_1x_2\cdots x_n}$ exists and also equals $l$.   
\end{proof}

\vspace{3mm}

\begin{proposition}
	If $x_n>0(n=1,2,\cdots)$ and the limit $\lim\limits_{n\to\infty}\dfrac{x_{n+1}}{x_n}$ exists, the limit $\lim\limits_{n\to\infty}\sqrt[^n]{x_n}$ also exists and
	\[
	\lim\limits_{n\to\infty}\sqrt[^n]{x_n}=\lim\limits_{n\to\infty}\frac{x_{n+1}}{x_n}.
	\]
\end{proposition}

\begin{proof}
    Assume $x_0=1$ and by Corollary 4.1.2 we immediately get
	\[
	\lim\limits_{n\to\infty}\sqrt[^n]{x_n}=\lim\limits_{n\to\infty}\sqrt[^n]{\frac{x_1}{x_0}\frac{x_2}{x_1}\cdots\frac{x_{n}}{x_{n-1}}}=\lim\limits_{n\to\infty}\frac{x_n}{x_{n-1}}=\lim\limits_{n\to\infty}\frac{x_{n+1}}{x_n}.
	\]
\end{proof}

\begin{example}
	Find the limit $\lim\limits_{n\to\infty}\dfrac{n}{\sqrt[^n]{n!}}$.
\end{example}

\section{Stolz–Cesàro theorem}

\begin{theorem}[Stolz–Cesàro theorem with form of $\dfrac{0}{0}$]
	Assume $\{a_n\}$ and $\{b_n\}$ are two infinitesimal sequences of real numbers and $\{a_n\}$ is a strictly decreasing. If
	\[
		\lim\limits_{n\to\infty}\frac{b_{n+1}-b_n}{a_{n+1}-a_n}=l,
	\]
    where $l$ is finite or $\pm\infty$, then we have
	\[
	    \lim\limits_{n\to\infty}\frac{b_n}{a_n}=l.
	\]
\end{theorem}

\begin{proof}
	Here we only prove it for a finite number $l$. According to the assumption, for every positive number $\epsilon$, there exists another positive number $N$ such that
	\[
	\left|\frac{b_{n+1}-b_n}{a_{n+1}-a_n}-l\right|<\epsilon
	\]
	for all $n>N$. Since $a_n>a_{n+1}$ for all $n\in\mathbb{N^*}$, we have
	\[
	(l-\epsilon)(a_n-a_{n+1})<b_n-b_{n+1}<(l+\epsilon)(a_n-a_{n+1}).
	\]
	Given $m>n$, replace $n$ by $n+1,n+2,\cdots,m-1$. Then add up the $m-n$ inequalities and we obtain
	\[
	(l-\epsilon)(a_n-a_m)<b_n-b_m<(l+\epsilon)(a_n-a_m),
	\]
	or
	\[
	\left|\frac{b_n-b_m}{a_n-a_m}-l\right|<\epsilon.
	\]
	Note that $\lim\limits_{m\to\infty}a_m=\lim\limits_{m\to\infty}b_m=0$. When $m\to\infty$, for all $n>N$ it follows that
	\[
	\left|\frac{b_n}{a_n}-l\right|\leqslant\epsilon.
	\]
	With the arbitrariness of selection of $\epsilon$, this implies $\lim\limits_{n\to\infty}\dfrac{b_n}{a_n}=l$.
\end{proof}

\vspace{3mm}

\begin{theorem}[Stolz–Cesàro theorem with form of $\dfrac{*}{\infty}$]
	Asume $\{a_n\}$ is a strictly increasing sequence such that $\lim a_n = \infty$. If
	\[
	\lim\limits_{n\to\infty}\frac{b_{n+1}-b_n}{a_{n+1}-a_n},
	\]
	where $l$ is finite or $\pm\infty$, then we have
	\[
	\lim\limits_{n\to\infty}\frac{b_n}{a_n}=l.
	\]
\end{theorem}

\begin{proof}
	We just consider a finite $l$. For every positive number $\epsilon$, there exists $N_1\in\mathbb{N^*}$ such that
	\[
	\left|\frac{b_{n+1}-b_n}{a_{n+1}-a_n}-l\right|<\epsilon
	\]
	for all $n>N_1$. Since $a_{n+1}>a_n$ for all $n\in\mathbb{N^*}$, we have
	\[
	(l-\epsilon)(a_{n+1}-a_{n})<b_{n+1}-b_{n}<(l+\epsilon)(a_{n+1}-a_{n}).
	\]
	Given $N_1$, replace $n$ by $N_1,N_1+1,\cdots,n-1$. Then add up the $n-N$ inequalities and we obtain
	\[
	(l-\epsilon)(a_n-a_{N_1})<b_n-b_{N_1}<(l+\epsilon)(a_n-a_{N_1}),
	\]
	or
	\[
	\left|\frac{b_n-b_{N_1}}{a_n-a_{N_1}}-l\right|<\epsilon.
	\]
	In order to estimate the value of $\left|\dfrac{b_n}{a_n}-l\right|$, consider the following identity
	\[
	\frac{b_n}{a_n}-l=\left(1-\frac{a_{N_1}}{a_n}\right)\left(\frac{b_n-b_{N_1}}{a_n-a_{N_1}}-l\right)+\frac{b_n-la_{N_1}}{a_n}.
	\]
	Since $\lim\limits_{n\to\infty}a_n=+\infty$, there exsits a positive number $N_2$ such that for all $n>N_2$
	\[
	0<\left|\frac{a_N}{a_n}\right|<1\Leftrightarrow0<1-\frac{a_N}{a_n}<2
	\]
	and
	\[
	\left|\frac{b_n-la_{N_1}}{a_n}\right|<\epsilon.
	\]
	Thus for all $n>\max\{N_1,N_2\}$ we have
	\[
	\left|\dfrac{b_n}{a_n}-l\right|<3\epsilon,
	\]
	which indicates $\lim\limits_{n\to\infty}\dfrac{b_n}{a_n}=l$.

\end{proof}




\section{Subsequence}

\begin{proposition}
    Every real sequence $\{a_n\}$ has a monotonic subsequence.
\end{proposition}

\begin{proof}
	Define $X=\{n\in \mathbb{N^*}| \;\forall k\ge n, a_k\ge a_n\}$, which is a subset of the index set of the sequence $\{a_n\}$.\\
	If $X$ is an infinite set, we can find an arrangement of the elements in $X$: $n_1<n_2<\cdots<n_i<\cdots(n_i\in X)$. Thus we get an increasing subsequence of $\{a_n\}$: $a_{n_1},a_{n_2},\cdots,a_{n_i},\cdots$.\\
	If $X$ is a finite set, denote $N=\max X$. For every $n>N$, there exists a number $k>n$ such that $a_k<a_n$. Let $n_1=N+1$ and there exists a number $n_2>n_1$ such that $a_{n_2}<a_{n_1}$. Since $n_2$ is also greater than $N$, there exists $n_3>n_2$ such that $a_{n_3}<a_{n_2}$. Repeating this process will come to a decreasing subsequence of $\{a_n\}$.
\end{proof}

\chapterimage{chapter_head_5.pdf}


\chapter{Limit of function}

\section{Equivalent Infinitesimal}

\begin{definition}
If the relation $f(x)=\gamma(x)g(x)$ holds ultimately over $\mathcal{B}$ where $\lim\limits_{\mathcal{B}}{\gamma(x)}=1$, we say that \emph{the function $f$ behaves asymptotically like $g$ over $\mathcal{B}$}, or, more briefly, that $f$ \emph{is equivalent to $g$ over $\mathcal{B}$}.



\end{definition}

%----------------------------------------------------------------------------------------
%	PART
%----------------------------------------------------------------------------------------

\part{Part N}


%----------------------------------------------------------------------------------------
%	BIBLIOGRAPHY
%----------------------------------------------------------------------------------------

\chapterimage{chapter_Bibliography.pdf}

\chapter*{Bibliography}
\addcontentsline{toc}{chapter}{\textcolor{ocre}{Bibliography}}
\section*{Books}
\addcontentsline{toc}{section}{Books}
\printbibliography[heading=bibempty,type=book]
\section*{Articles}
\addcontentsline{toc}{section}{Articles}
\printbibliography[heading=bibempty,type=article]

%----------------------------------------------------------------------------------------
%	INDEX
%----------------------------------------------------------------------------------------


\chapterimage{chapter_Index.pdf}
\cleardoublepage
\phantomsection
\setlength{\columnsep}{0.75cm}
\addcontentsline{toc}{chapter}{\textcolor{ocre}{Index}}
\printindex

%----------------------------------------------------------------------------------------

\end{document}
